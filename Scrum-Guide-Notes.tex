\documentclass[a4paper,11pt,twocolumn]{article}

\usepackage[utf8]{inputenc}
\usepackage[left=1.8cm, right=1.8cm, top=3cm, bottom=3cm]{geometry}
\usepackage{graphicx}
\usepackage[hidelinks]{hyperref}
\usepackage{titlesec}
\usepackage{enumitem}
\usepackage{fancyhdr}
\usepackage{lastpage}
\usepackage{url}

% Set page numbers to show "Page X of Y"
\pagestyle{fancy}
\fancyhf{}
\fancyhead{}
\renewcommand{\headrulewidth}{0pt}
\rfoot{Page \thepage \hspace{1pt} of \pageref{LastPage}}

% Change section format
\titleformat{\section}
  {\normalfont\large\bfseries}{\thesection}{1em}{}[{\vspace{0.3em}\titlerule[2pt]\vspace{0em}}]

% Change list format
\setlist[1]{itemsep=-2pt}
\setlist[itemize,1]{leftmargin=*}

% Change list icon to simple dash
\renewcommand\labelitemi{-}

% Set column spacing
\setlength{\columnsep}{1.5cm}

% Change paragraph identation
\parindent 0mm

\title{\textbf{Scrum Guide Notes} \vspace{-1ex}}
\author{Vasileios Papadopoulos}
\date{}

\begin{document}

\maketitle

% Set first page style to "fancy" as it will display a different page number by default
\thispagestyle{fancy}

\section*{Definition of Scrum}
\textit{Scrum (n): A framework within which people can address complex adaptive problems, while productively and creatively delivering products of the highest possible value.}

\section*{Scrum Characteristics}
\begin{itemize}
	\item Lightweight
	\item Simple to understand
	\item Difficult to master
\end{itemize}

\section*{Empiricism}
\textit{Knowledge comes from experience and making decisions on what is known.}

\section*{Scrum Consists of}
\begin{itemize}
	\item Roles
	\item Events
	\item Artifacts
	\item Rules
\end{itemize}

\section*{Scrum Optimizes}
\begin{itemize}
	\item Flexibility
	\item Creativity
	\item Productivity
\end{itemize}

\section*{Three Pillars of Scrum}
\begin{itemize}
	\item \textbf{Transparency}\\
	Make significant aspects of the process visible to those responsible for the outcome
	\item \textbf{Inspection}\\
		Frequently inspect the progress towards a goal to detect undesirable variances
	\item \textbf{Adaption}\\
	Adjust the process as soon as possible to minimize further deviation
\end{itemize}

\section*{Scrum Values}
\begin{itemize}
    \item \textbf{Commitment}\\
    People personally commit to achieving the goals of the Scrum Team
    \item \textbf{Courage}\\
    The Scrum Team members have courage to do the right thing and work on tough problems
    \item \textbf{Focus}\\
    Everyone focuses on the work of the Sprint and the goals of the Scrum Team
    \item \textbf{Openness}\\
    The Scrum Team and its stakeholders agree to be open about all the work and the challenges with performing the work
    \item \textbf{Respect}\\
    Scrum Team members respect each other to be capable, independent people
\end{itemize}

\section*{The Scrum Team}
\begin{itemize}
    \item Product Owner
    \item Development Team
    \item Scrum Master
\end{itemize}

\section*{Scrum Team Characteristics}
\begin{itemize}
	\item Deliver products \textit{iteratively} and \textit{incrementally}
	\item Self-organizing
	\item Cross-functional
\end{itemize}

\section*{Self-organization}
\textit{Self-organizing teams choose how best to accomplish their work, rather than being directed by others outside the team.}

\section*{Cross-functionality}
\textit{Cross-functional teams have all competencies needed to accomplish the work without depending on others not part of the team.}

\section*{The Product Owner}
\begin{itemize}
    \item Responsible for maximizing the value of the product resulting from work of the development team
    \item The \textbf{only} person responsible for managing the Product Backlog
	\item \textbf{One person}, not a committee
	\item His/her decisions are visible in the content and ordering of the Product Backlog
	\item The Product Owner may have the Development Team manage the Product Backlog however, they remain \textbf{accountable}
\end{itemize}

\section*{Product Backlog Management}
\begin{itemize}
    \item Clearly expressing Product Backlog Items
    \item Ordering the items in the Product Backlog to best achieve goals and missions
    \item Optimizing the value of work the Development Team performs
    \item Ensuring that the Product Backlog is visible, transparent, clear to all and shows what the Scrum Team will work on next
    \item Ensuring that the Development Team understands the Product Backlog to the level needed
\end{itemize}

\section*{The Development Team}
\begin{itemize}
	\item Delivers a \textit{potentially releasable} Increment of ``Done'' product at the end of each Sprint 
	\item Only members of the Development Team create the Increment
    \item Size: 3 to 9 members
\end{itemize}

\section*{Development Team Characteristics}
\begin{itemize}
    \item Self-organizing
    \item Cross-functional
	\item Scrum recognizes no titles for Development Team members
	\item Scrum recognizes no sub-teams in the Development Team
	\item Accountability belongs \textbf{to the team} as a whole
\end{itemize}

\section*{The Scrum Master}
\begin{itemize}
    \item Promotes and supports Scrum
    \item Helps everyone understand Scrum theory, practices and rules 
    \item Acts as a \textit{Servant-Leader} for the Scrum Team
    \item Helps those outside the Scrum Team understand which interactions are helpful and which aren't
\end{itemize}

\section*{Scrum Master - Product Owner}
\begin{itemize}
    \item Ensure that goals, scope, and product domain are understood by everyone on the Scrum Team as well as possible
    \item Find techniques for effective Product Backlog management
    \item Help the Scrum Team understand the need for clear and concise Product Backlog Items
    \item Understand product planning in an empirical environment
    \item Ensuring the Product Owner knows how to arrange the Product Backlog to maximize value
    \item Understanding and practicing agility
    \item Facilitating Scrum events as requested or needed
\end{itemize}

\section*{Scrum Master - Development Team}
\begin{itemize}
    \item Coaches the Development Team in self-organization and cross-functionality
    \item Helps the Development Team to create high-value products
    \item Removes impediments to the Development Team's progress
    \item Facilitates Scrum events as requested or needed
    \item Coaches the Development Team in organizational environments in which Scrum is not yet fully adopted and understood
\end{itemize}

\section*{Scrum Master - Organization}
\begin{itemize}
    \item Leads and coaches the organization in its Scrum adoption
    \item Plans Scrum implementations within the organization
    \item Helps employees and stakeholders understand and enact Scrum and empirical product development
    \item Causes change that increases the productivity of the Scrum Team
    \item Works with other Scrum Masters to increase the effectiveness of the application of Scrum in the organization
\end{itemize}

\section*{Scrum Events}
\begin{itemize}
    \item The Sprint
	\item Sprint Planning
	\item Daily Scrum
	\item Sprint Review
	\item Sprint Retrospective
\end{itemize}

\section*{Scrum Event Characteristics}
\begin{itemize}
    \item Create regularity
	\item Minimize the need for meetings not defined in Scrum
	\item Time-boxed (Have max duration)
	\item Each event is an opportunity to inspect and adapt something
	\item Failure to include any of these events results in reduced transparency
\end{itemize}

\section*{The Sprint}
\begin{itemize}
    \item Acts as a container for all other events
	\item Duration: One month or less (consistent duration)
	\item A new sprint starts immediately after the conclusion of the previous Sprint
	\item A ``Done'', potentially releasable Product Increment is created
	\item Consists of:
	\vspace{-1em}
    \begin{itemize}
        \setlength\itemsep{0em}
        \item Sprint Planning
        \item Daily Scrums
        \item Development Work
        \item Sprint Review
        \item Sprint Retrospective
    \end{itemize}
\end{itemize}

\section*{During the Sprint}
\begin{itemize}
    \item No changes are made that would endanger the Sprint Goal
	\item Quality goals do not decrease
	\item Scope may be clarified and re-negotiated between the Product Owner and the Development Team as more is learned
\end{itemize}

\section*{Cancelling a Sprint}
\begin{itemize}
    \item \textbf{Only} the Product Owner has the authority to cancel a Sprint
	\item The Sprint is cancelled if the Sprint Goal becomes \textit{obsolete}
	\item Any completed and ``Done'' Product Backlog items are reviewed
    \item If part of the work is potentially releasable, the Product Owner typically accepts it
    \item All incomplete Product Backlog Items are re-estimated and put back on the Product Backlog
\end{itemize}

\section*{Sprint Planning}
\begin{itemize}
    \item What can be delivered in the Increment resulting from the upcoming Sprint?
	\item How will the work needed to deliver the Increment be achieved?
	\item The plan is created by the collaborative work of the entire Scrum Team
	\item Max duration: 8 hours for one-month Sprint
	\item Attendees: All Scrum Team members
\end{itemize}

\section*{Sprint Planning - Input}
\begin{itemize}
    \item Product Backlog
    \item Latest Product Increment
    \item Projected capacity of the Development Team during the Sprint
    \item Past Performance of the Development Team
\end{itemize}

\section*{Sprint Planning - Notes}
\begin{itemize}
    \item The number of items selected from the Product Backlog for the Sprint is \textbf{solely} up to the Development Team
	\item The Product Owner can help to clarify selected Product Backlog Items and make trade-offs
	\item The Development Team may renegotiate selected Product Backlog Items with the Product Owner
	\item The Development Team may invite other people to attend to provide technical or domain advice
	\item Output: \textit{Sprint Backlog} and \textit{Sprint Goal}
\end{itemize}

\section*{Sprint Backlog}
\begin{itemize}
    \item The Product Backlog Items selected for this Sprint
    \item A plan for delivering them
\end{itemize}

\section*{Sprint Goal}
\begin{itemize}
    \item An objective that will be met within the Sprint through the implementation of the selected Product Backlog Items
    \item Provides guidance to the Development Team on \textit{why} it is building the Increment
\end{itemize}

\section*{Daily Scrum}
\begin{itemize}
    \item Held every day of the Sprint at the same place and time
    \item The Development Team plans work for the next 24 hours
	\item Max duration: 15 minutes
	\item Attendees: All Development Team members
\end{itemize}

\section*{Daily Scrum - Notes}
\begin{itemize}
	\item The Scrum Master ensures that the Development Team has the meeting but the Development Team is \textbf{responsible for conducting} the Daily Scrum
	\item The Scrum Master teaches the team to keep the Daily Scrum within the 15-minute time-box
	\item The Development Team or team members often meet immediately after the Daily Scrum for related discussions
    \item It is an internal meeting for the Development Team. If others are present, the Scrum Master ensures they \textbf{do not disrupt} the meeting
\end{itemize}

\section*{Daily Scrum - Benefits}
\begin{itemize}
	\item Improve communications
	\item Eliminate other meetings
	\item Identify impediments to development for removal
	\item Highlight and promote quick decision-making
	\item Improve the Development Team's level of knowledge
\end{itemize}

\section*{Sprint Review}
\begin{itemize}
	\item Held at the end of each Sprint
	\item Inspect the Increment and adapt the Product Backlog if needed
	\item Collaborate on the next things that could be done to optimize value
	\item Max duration: 4 hours for a one-month Sprint
	\item Attendees: All Scrum Team members and Stakeholders\\
	(Invited by the Product Owner)
\end{itemize}

\section*{Sprint Review - Notes}
\begin{itemize}
	\item The Sprint Review is \textbf{not a demo}
	\item The presentation of the Increment is intended to elicit feedback and foster collaboration
	\item Result: A revised Product Backlog that defines the \textit{probable} Product Backlog Items for the next Sprint
\end{itemize}

\section*{Sprint Review - Scrum Master}
\begin{itemize}
	\item Ensures that the event takes place
	\item Everyone understands its purpose
	\item Teaches everyone to keep it within the time-limit
\end{itemize}

\section*{Sprint Retrospective}
\begin{itemize}
	\item Occurs after the Sprint Review and prior to the next Sprint Planning
	\item Identify how the last Sprint went with regards to people, relationships, processes and tools
	\item Identify and order the major items that went well and potential improvements
	\item Create a \textit{plan} fir implementing improvements to the way the Scrum Team does its work
	\item Max duration: 3 hours for a one-month Sprint
	\item Attendees: All Scrum Team members
\end{itemize}

\section*{Sprint Retrospective - Scrum Master}
\begin{itemize}
	\item Ensures that the event takes place and attendands understand its purpose
	\item Ensures that the meeting is \textit{positive} and \textit{productive}
	\item Teaches everyone to keep it within the time-box
	\item Participates as peer team member from the accountability ove the Scrum process
	\item Encourages the Scrum Team to improve
\end{itemize}

\textit{Note: During each Sprint Retrospective, the Scrum Team plans ways to increase product quality by improving work processes or adapting the definition of ``Done'' if appropriate and not in conflict with the product or organizational standards}

\section*{Scrum Artifacts}
\begin{itemize}
	\item Product Backlog
	\item Sprint Backlog
	\item Increment
\end{itemize}

\section*{Product Backlog}
\begin{itemize}
	\item Ordered list of everything known to be needed in the Product
	\item Single source of requirements for any changes to be made to the Product
	\item It is dynamic (evolves)
	\item It is never complete
	\item If a Product exists, a Product Backlog exists too
\end{itemize}

\textit{Note: If multiple Scrum Teams work on the same Product, only one Product Backlog is used}

\section*{Product Backlog Items - Attributes}
\begin{itemize}
	\item Description
    \item Order
    \item Estimate
    \item Value
    \item May also include test descriptions that will prove the item's completeness when ``Done''
\end{itemize}

\section*{Product Backlog Refinement}
\begin{itemize}
	\item The act of adding \textbf{detail}, \textbf{estimates}, and \textbf{order} to items in the Product Backlog
	\item The Product Owner and the Development Team cooperate during refinement
	\item The Scrum Team decides \textit{how} and \textit{when} refinement is done
	\item \textit{Usually} consumes no more than 10\% of the Development Team's capacity
\end{itemize}

\section*{Product Backlog Refinement - Notes}
\begin{itemize}
	\item Product Backlog Items can be updated anytime by the Product Owner or at the Product Owner's discretion
	\item \textbf{Only} the Development Team is responsible for any estimates
\end{itemize}

\section*{Sprint Backlog}
\begin{itemize}
	\item The set of Product Backlog Items selected for the Sprint
	\item A plan for delivering the Increment and realizing the Sprint Goal
	\item Includes at least one high priority process improvement identified in the previous Retrospective meeting
	\item Belongs \textbf{solely} to the Development Team
	\item Only the Development Team can change the Sprint Backlog during a Sprint
\end{itemize}

\section*{Increment}
\begin{itemize}
	\item The sum of all Product Backlog Items completed during the Sprint and the value of Increments of all previous Sprints
	\item A step towards a vision or a goal
	\item Must be in \textit{usable condition} regardless of whether the Product Owner decides to release it
\end{itemize}

\section*{Definition of ``Done''}
\begin{itemize}
	\item When a Product Backlog Item or an Increment is described as ``Done'', everyone must understand what ``Done'' means
	\item The definition of ``Done'' is used to assess when work is complete on the product Increment
    \item It also guides the Development Team in knowing how many Product Backlog Items it can select during a Sprint Planning
\end{itemize}

\section*{Definition of ``Done'' - Notes}
\begin{itemize}
    \item Multiple Scrum Teams working on the same Product should have the same definition of ``Done''
    \item If the definition of ``Done'' is part of the conventions, standards or guidelines of the development organization, all Scrum Teams \textbf{must} follow it as a minimum
	\item As Scrum Teams mature, it is expected that their definitions of ``Done'' will expand to include more stringent criteria for higher quality
\end{itemize}

\nocite{*}
\bibliographystyle{plain}
\bibliography{bibliography}

\end{document}