\documentclass[a4paper,11pt,twocolumn]{article}

\usepackage[utf8]{inputenc}
\usepackage[left=1.8cm, right=1.8cm, top=2.8cm, bottom=2.8cm]{geometry}
\usepackage{graphicx}
\usepackage[hidelinks]{hyperref}
\usepackage{titlesec}
\usepackage{enumitem}
\usepackage{fancyhdr}
\usepackage{lastpage}
\usepackage{tikz}
\usepackage[most]{tcolorbox}
\usepackage{url}

% Set page numbers to show "Page X of Y"
\pagestyle{fancy}
\fancyhf{}
\fancyhead{}
\renewcommand{\headrulewidth}{0pt}
\rfoot{Page \thepage \hspace{0.5pt} of \pageref{LastPage}}

% Change section format
\titleformat{\section}
  {\normalfont\large\bfseries}{\thesection}{1em}{}[{\vspace{0.3em}\titlerule[2pt]\vspace{0em}}]

% Change list format
\setlist[1]{itemsep=-2pt}
\setlist[itemize,1]{leftmargin=*}
\setlist[enumerate,1]{leftmargin=*}
\setlist[itemize,2]{label=$\circ$,leftmargin=1.1em}

% Change list icon to simple dash
\renewcommand\labelitemi{-}

% Set column spacing
\setlength{\columnsep}{1.5cm}

% Change paragraph identation
\parindent 0mm

\title{\textbf{Scrum Guide Notes} \vspace{-2ex}}
\author{Vasileios Papadopoulos}
\date{}

\begin{document}

\maketitle

% Set first page style to "fancy" as it will display a different page number by default
\thispagestyle{fancy}

\section*{Agile Software Development}
\begin{itemize}
	\item A different approach to the software development process
	\item Focuses on the clean delivery of individual pieces or parts of the software and \textbf{not} on the entire application
	\item Requirements \& solutions evolve through the collaborative effort of teams and their customers and/or end users
	\item Encourages \textit{early delivery} and \textit{continuous improvement}
	\item The term Agile was popularized in 2001 when the \textit{Agile Manifesto} was published
\end{itemize}

\section*{Agile Manifesto}
\begin{itemize}
	\item \textbf{Individuals and interactions}\\
	      over processes and tools
	\item \textbf{Working software}\\
	      over comprehensive documentation
	\item \textbf{Customer collaboration}\\
	      over contract negotiation
	\item \textbf{Responding to change}\\
	      over following a plan
\end{itemize}

\section*{Definition of Scrum}
\textit{A lightweight \textbf{framework} that helps people, teams and organizations generate value through adaptive solutions for complex problems.}

\section*{What is a framework?}
A framework is a system of rules, ideas, or beliefs that is used to plan or decide something.

\begin{tcolorbox}[colback=black!8!white,colframe=gray!50!black,title=Note,sharp corners,fonttitle=\normalsize\bfseries,fontupper=\normalsize,left=0.7em,right=0.7em]
	As a framework, Scrum will never provide you detailed instructions or show you exactly how to deal with problems. Instead, it provides a set of rules that guide people's relationships and interactions
\end{tcolorbox}

\section*{Concept}
A Scrum Master creates an environment where:
\begin{enumerate}
	\item A Product Owner orders the work for a complex problem into a Product Backlog
	\item The Scrum Team turns a selection of the work into an Increment of value during a Sprint
	\item The Scrum Team and its stakeholders inspect the results and adjust for the next Sprint
	\item \textit{Repeat}
\end{enumerate}

\section*{Scrum Characteristics}
\begin{itemize}
	\item Simple
	\item Purposefully incomplete
	\item Built upon by the \textit{collective intelligence} of the people using it
	\item Employs an iterative, incremental approach to \textit{optimize predictability} and \textit{control risk}
	\item Makes visible the relative efficacy of current management, environment and work techniques, so that \textit{improvements} can be made
\end{itemize}

\section*{Foundations}
\begin{itemize}
	\item \textbf{Empiricism}\\
	      \textit{Knowledge comes from experience and making decisions based on what is observed}
	\item \textbf{Lean thinking}\\
	      \textit{Reduce waste (non-value added activities) and focus on the essentials}
\end{itemize}

\section*{Three Pillars of Scrum}
\begin{itemize}
	\item \textbf{Transparency}\\
	      Make the emergent process and work visible to those performing the work as well as those receiving the work
	\item \textbf{Inspection}\\
	      Frequently inspect the Scrum artifacts and progress toward agreed goals to detect potentially undesirable variances or problems
	\item \textbf{Adaptation}\\
	      If any aspects of a process deviate outside acceptable limits or if the resulting product is unacceptable adjustments must be made as soon as possible to minimize further deviation
\end{itemize}

\section*{Scrum Values}
\begin{itemize}
	\item \textbf{Commitment}\\
	      The Scrum Team commits to achieving its goals and supporting each other
	\item \textbf{Focus}\\
	      Everyone focuses on Sprint work to make the best possible progress toward the goals
	\item \textbf{Openness}\\
	      The Scrum Team and its stakeholders are open about the work and the challenges
	\item \textbf{Respect}\\
	      Scrum Team members respect each other to be capable, independent people. They are also respected as such by the people with whom they work
	\item \textbf{Courage}\\
	      Scrum Team members have the courage to do the right thing and work on tough problems
\end{itemize}

\section*{The Scrum Team}
\begin{itemize}
	\item \textit{One} Scrum Master
	\item \textit{One} Product Owner
	\item Developers
\end{itemize}

\section*{Scrum Team Characteristics}
\begin{itemize}
	\item \textit{Typically} 10 or fewer people
	\item No sub-teams or hierarchies
	\item Focused on \textit{one} objective at a time, the Product Goal
	\item \textit{Accountable} for creating a valueable, useful Increment every Sprint
	\item Self-managing
	\item Cross-functional
\end{itemize}

\begin{tcolorbox}[colback=black!8!white,colframe=gray!50!black,title=Note,sharp corners,fonttitle=\normalsize\bfseries,fontupper=\normalsize,left=0.7em,right=0.7em]
	The Scrum Guide \textit{suggests} to have a team of up to 10 members but this is not a \textit{requirement}. A Scrum Team can still have 15 members however, it won't be as effective.
\end{tcolorbox}

\section*{Self-management}
\textit{Self-managing teams internaly decide who does \textbf{what}, \textbf{when} and \textbf{how}.}

\section*{Cross-functionality}
\textit{Cross-functional teams have all the skills necessary to create value each Sprint.}

\hspace{0.5pt}

\begin{tcolorbox}[colback=black!8!white,colframe=gray!50!black,title=Note,sharp corners,fonttitle=\normalsize\bfseries,fontupper=\normalsize,left=0.7em,right=0.7em]
	If a Scrum Team becomes too large, consider reorganizing into multiple cohesive Scrum Teams that work on the same product and share the same Product Goal, Product Backlog and Product Owner
\end{tcolorbox}

\section*{Developers}
\begin{itemize}
	\item The people in the Scrum Team that are committed to creating any aspect of a \textit{usable} Increment each Sprint
\end{itemize}
\textbf{Accountable for:}
\begin{itemize}
	\item Creating a plan for the Sprint - the Sprint Backlog
	\item Instilling quality by adhering to a Definition of Done
	\item Adapting their plan each day toward the Sprint Goal
	\item Holding each other accountable as professionals
\end{itemize}

\section*{The Product Owner}
\begin{itemize}
	\item Accountable for maximizing the value of the product resulting from the work of the Scrum Team
	\item Accountable for effective Product Backlog management
	\item One person, not a committee
	\item Their decisions are visible in the content and ordering of the Product Backlog and through the inspectable Increment at the Sprint Review
	\item The \textit{entire organization} must respect their decisions
\end{itemize}

\section*{Product Backlog Management}
\begin{itemize}
	\item Develop and explicitly communicate the \textit{Product Goal}
	\item Create and clearly communicate Product Backlog items
	\item Order Product Backlog items
	\item Ensure that the Product Backlog is transparent, visible and understood
\end{itemize}

\begin{tcolorbox}[colback=black!8!white,colframe=gray!50!black,title=Note,sharp corners,fonttitle=\normalsize\bfseries,fontupper=\normalsize,left=0.7em,right=0.7em]
	The Product Owner may do the above work or may delegate the responsibility to others. They, however, remain \textbf{accountable}
\end{tcolorbox}

\section*{The Scrum Master}
\begin{itemize}
	\item Accountable for establishing Scrum as defined in the Scrum Guide
	\item Helps everyone understand Scrum theory, practices and rules both within the Scrum Team and the organization
	\item Accountable for the Scrum Team's effectiveness
	\item \textit{Leader who serves} the Scrum Team and the larger organization
\end{itemize}

\begin{tcolorbox}[colback=black!8!white,colframe=gray!50!black,title=Note,sharp corners,fonttitle=\normalsize\bfseries,fontupper=\normalsize,left=0.7em,right=0.7em]
	As a leader, the Scrum Master guides and coaches the Scrum Team to improve its practices within the Scrum framework
\end{tcolorbox}

\section*{Scrum Master - Scrum Team}
\begin{itemize}
	\item Coaches the team members in self-management and cross-functionality
	\item Helps the Scrum Team focus on creating high-value Increments that meet the Definition of Done
	\item Causes the removal of impediments to the Scrum Team's progress
	\item Ensures that all Scrum events take place and are \textit{positive}, \textit{productive} and kept within the \textit{timebox}
\end{itemize}

\section*{Scrum Master - Product Owner}
\begin{itemize}
	\item Helps find techniques for effective Product Goal definition and Product Backlog management
	\item Helps the Scrum Team understand the need for clear and concise Product Backlog items
	\item Helps establish empirical product planning for a complex environment
	\item Facilitates stakeholder collaboration as requested or needed
\end{itemize}

\section*{Scrum Master - Organization}
\begin{itemize}
	\item Leads, trains and coaches the organization in its Scrum adoption
	\item Plans and advises Scrum implementations within the organization
	\item Helps employees and stakeholders understand and enact an empirical approach for complex work
	\item Removes barriers between stakeholders and Scrum Teams
\end{itemize}

\section*{Scrum Events}
\begin{itemize}
	\item The Sprint
	\item Sprint Planning
	\item Daily Scrum
	\item Sprint Review
	\item Sprint Retrospective
\end{itemize}

\section*{Scrum Event Characteristics}
\begin{itemize}
	\item Create regularity
	\item Minimize the need for meetings not defined in Scrum
	\item Designed to enable the transparency required
	\item Each event is a formal opportunity to inspect and adapt Scrum artifacts
	\item Failure to operate any events as prescribed results in lost opportunities to inspect and adapt
	\item \textit{Optimally}, all events are held at the same time and place to reduce complexity
\end{itemize}

\section*{The Sprint}
\begin{itemize}
	\item Acts as a container for all other events
	\item Duration: One month or less (consistent duration)
	\item A new sprint starts immediately after the conclusion of the previous Sprint
	\item Sprints enable predictability by ensuring inspection and adaptation of progress toward a Product Goal \textit{at least} every calendar month
	\item Each Sprint may be considered a short project
\end{itemize}

\begin{tcolorbox}[colback=black!8!white,colframe=gray!50!black,title=Note,sharp corners,fonttitle=\normalsize\bfseries,fontupper=\normalsize,left=0.7em,right=0.7em]
	When a Sprint's horizon is too long the Sprint Goal may become invalid, complexity may rise and risk may increase. Shorter Sprints can be employed to generate more learning cycles and limit risk of cost and effort to a smaller time frame
\end{tcolorbox}

\section*{During the Sprint}
\begin{itemize}
	\item No changes are made that would endanger the Sprint Goal
	\item Quality does not decrease
	\item The Product Backlog is refined as needed
	\item Scope may be clarified and renegotiated with the Product Owner as more is learned
\end{itemize}

\section*{Cancelling a Sprint}
\begin{itemize}
	\item \textbf{Only} the Product Owner has the authority to cancel a Sprint
	\item A Sprint could be cancelled if the Sprint Goal becomes \textit{obsolete}
\end{itemize}

\section*{Sprint Planning}
\begin{itemize}
	\item Initiates the Sprint by laying out the work to be performed for the Sprint
	\item Topics:
	      \begin{itemize}
	      	\item Why is this Sprint valuable?
	      	\item What can be Done this Sprint?
	      	\item How will the chosen work get done?
	      \end{itemize}
	\item The resulting plan is created by the collaborative work of the \textit{entire} Scrum Team
	\item Max duration: 8 hours for one-month Sprint
	\item Attendees: All Scrum Team members
\end{itemize}

\section*{Sprint Planning - Notes}
\begin{itemize}
	\item The Product Owner ensures that attendees are \textit{prepared} to discuss the most important Product Backlog items and how they map to the Product Goal
	\item The whole Scrum Team collaborates to define a Sprint Goal that communicates why the Sprint is valuable to stakeholders
	\item Developers will become more confident in their Sprint forecasts as they learn more about their performance, upcoming capacity and the Definition of Done
	\item The Scrum Team may refine Product Backlog items during planning to increase its understanding and confidence
	\item How Developers plan to turn Product Backlog items to an Increment that meets the Definition of Done is at their \textbf{solely} up to them
	\item The Scrum Team may invite other people to attend the Sprint Planning to provide advice
	\item Output: \textit{Sprint Backlog}
\end{itemize}

\section*{Sprint Backlog}
\begin{itemize}
	\item The Sprint Goal
	\item The Product Backlog items selected for the Sprint
	\item A plan for delivering them
\end{itemize}

\section*{Daily Scrum}
\begin{itemize}
	\item Inspect progress toward the Sprint Goal and adapt the Sprint Backlog as necessary
	\item Held every working day of the Sprint at the same place and time
	\item Max duration: 15 minutes
	\item Attendees: All Developers
\end{itemize}

\section*{Daily Scrum - Notes}
\begin{itemize}
	\item If the Product Owner or Scrum Master are actively working on items in the Sprint Backlog, they participate as Developers
	\item The Developers can select whatever structure and techniques they want, as long as their Daily Scrum focuses on progress toward the Sprint Goal and produces an actionable plan for the next day of work
\end{itemize}

\section*{Daily Scrum - Benefits}
\begin{itemize}
	\item Improve communication
	\item Identify impediments
	\item Promote quick decision-making
	\item Eliminate the need for other meetings
\end{itemize}

\begin{tcolorbox}[colback=black!8!white,colframe=gray!50!black,title=Note,sharp corners,fonttitle=\normalsize\bfseries,fontupper=\normalsize,left=0.7em,right=0.7em]
	The Daily Scrum is \textbf{not} the only time Developers are allowed to adjust their plan. They often meet throughout the day for more detailed discussions about adapting or re-planning the rest of the Sprint's work
\end{tcolorbox}

\section*{Sprint Review}
\begin{itemize}
	\item Inspect the outcome of the Sprint and determine future adaptations
	\item Held at the end of the Sprint, before the Sprint Retrospective
	\item Max duration: 4 hours for a one-month Sprint
	\item Attendees: All Scrum Team members and Stakeholders
\end{itemize}

\section*{Sprint Review - Notes}
\begin{itemize}
	\item The Sprint Review is a \textit{working session} and the Scrum Team should avoid limiting it to a presentation
	\item The Scrum Team presents the results of their work to key stakeholders and progress toward the Product Goal is discussed
	\item Attendees then examine what has changed in their environment and collaborate on what to do next
	\item The Product Backlog may then be adjusted to meet new opportunities
\end{itemize}

\section*{Sprint Retrospective}
\begin{itemize}
	% \item Plan ways to increase quality and effectiveness
	\item Inspect how the last Sprint went with regards to individuals, interactions, processes, tools, and the Definition of Done
	\item Discuss what went well during the Sprint, what problems were encountered and how those problems were (or were not) solved
	\item Identify the most helpful changes to improve effectiveness and \textit{address} them as soon as possible
	\item Last event of the Sprint
	\item Max duration: 3 hours for a one-month Sprint
	\item Attendees: All Scrum Team members
\end{itemize}


\section*{Sprint Retrospective - Notes}
\begin{itemize}
	\item Inspected elements often vary with the domain of work
	\item Identified improvements \textbf{may} be added to the Sprint Backlog for the next Sprint
\end{itemize}

\section*{Scrum Artifacts}
\begin{itemize}
	\item \textbf{Product Backlog}\\
	\textit{Commitment: Product Goal}
	\item \textbf{Sprint Backlog}\\
	\textit{Commitment: Sprint Goal}
	\item \textbf{Increment}\\
	\textit{Commitment: Definition of Done}
\end{itemize}

\begin{tcolorbox}[colback=black!8!white,colframe=gray!50!black,title=Note,sharp corners,fonttitle=\normalsize\bfseries,fontupper=\normalsize,left=0.7em,right=0.7em]
	Scrum artifacts represent work or value. Each artifact contains a commitment to ensure it provides information that enhances transparency and focus, against which progress can be \textbf{measured}
\end{tcolorbox}

\section*{Product Definition}
\textit{A product is a vehicle to deliver value. It has a clear boundary, known stakeholders, well-defined users or customers. A product could be a service, a physical product or something more abstract}

\section*{Product Backlog}
\begin{itemize}
	\item Emergent, ordered list of what is needed to improve the product
	\item Single source of work undertaken by the Scrum Team
	\item Product Backlog items that can be Done by the Scrum Team within one Sprint are deemed \textit{ready} for selection in a Sprint Planning event
	\item Product Backlog items \textit{usually} acquire the necessary degree of transparency after refining activities
\end{itemize}

\section*{Product Backlog Refinement}
\begin{itemize}
	\item The act of breaking down and further defining Product Backlog items into smaller, more precise items
	\item An ongoing activity during which, details are added to Product Backlog items such as a \textit{description}, \textit{order} and \textit{size}
	\item Added attributes vary with the domain of work
\end{itemize}

\begin{tcolorbox}[colback=black!8!white,colframe=gray!50!black,title=Note,sharp corners,fonttitle=\normalsize\bfseries,fontupper=\normalsize,left=0.7em,right=0.7em]
	The Developers who will be doing the work are responsible for sizing the corresponding Product Backlog items. Although the Product Owner may influence them by helping them understand and select trade-offs, the people who will perform the work make the final estimate
\end{tcolorbox}

\section*{Product Goal}
\begin{itemize}
	\item Describes a future state of the product 
	\item Serves as a long-term objective for the Scrum Team to plan against
	\item Resides in the Product Backlog. The rest of the Product Backlog emerges to define ``what" will fulfill the Product Goal
	\item The Scrum Team must fulfill (or abandon) one Product Goal before taking on the next
\end{itemize}

\section*{Sprint Backlog}
\begin{itemize}
	\item A plan \textbf{by and for} the Developers
	\item Consists of:
	\begin{itemize}
	\item The Sprint Goal (why)
	\item The set of Product Backlog items selected for the Sprint (what)
	\item An actionable plan for delivering the Increment (how)
	\end{itemize}
	\item Highly visible, \textit{real-time} picture of the work that the Developers plan to accomplish during the Sprint in order to achieve the Sprint Goal
	\item Updated throughout the Sprint as more is learned
	\item Should have enough detail so progress can be inspected during the Daily Scrum
\end{itemize}


\section*{Sprint Goal}
\begin{itemize}
	\item Single objective for the Sprint
	\item Creates coherence and focus
	\item Encourages the Scrum Team to work together rather than on separate initiatives
	\item Provides flexibility to the Developers in terms of the exact work needed to achieve it
\end{itemize}

\begin{tcolorbox}[colback=black!8!white,colframe=gray!50!black,title=Note,sharp corners,fonttitle=\normalsize\bfseries,fontupper=\normalsize,left=0.7em,right=0.7em]
	As the Developers work during the Sprint, they keep the Sprint Goal in mind. If the work turns out to be different than they expected, they collaborate with the Product Owner to negotiate the scope of the Sprint Backlog within the Sprint \textbf{without} affecting the Sprint Goal.
\end{tcolorbox}

\section*{Increment}
\begin{itemize}
	\item A concrete step toward the Product Goal
	\item Each Increment is additive to all prior Increments and thoroughly verified to ensure that all Increments work together
	\item Multiple Increments may be created within a Sprint
	\item Must be \textbf{usable}
	\item Work cannot be considered part of an Increment unless it meets the Definition of Done
\end{itemize}

\begin{tcolorbox}[colback=black!8!white,colframe=gray!50!black,title=Note,sharp corners,fonttitle=\normalsize\bfseries,fontupper=\normalsize,left=0.7em,right=0.7em]
	The Scrum Team has the ability to deliver an Increment to the stakeholders prior to the end of the Sprint. The Sprint Review should \textbf{never} be considered a gate to releasing value
\end{tcolorbox}

\section*{Definition of Done}
\begin{itemize}
	\item A formal description of the state of the Increment when it meets the quality measures required for the product
	\item Creates transparency by providing everyone a shared understanding of what work was completed as part of the Increment
	\item Developers are required to conform to the Definition of Done
	\item The moment a Product Backlog item meets the Definition of Done, an Increment is born
\end{itemize}

\begin{tcolorbox}[colback=black!8!white,colframe=gray!50!black,title=Note,sharp corners,fonttitle=\normalsize\bfseries,fontupper=\normalsize,left=0.7em,right=0.7em]
	If a Product Backlog item does not meet the Definition of Done by the end of the Sprint, it cannot be released or even presented at the Sprint Review. Instead, it returns to the Product Backlog for future consideration
\end{tcolorbox}

\section*{Definition of Done - Notes}
\begin{itemize}
	\item Multiple Scrum Teams working together on a Product must \textit{mutually} define and comply with the same Definition of Done
	\item If the Definition of Done is part of the standards of the organization, all Scrum Teams must follow it as a \textit{minimum}
	\item If there is not an organizational standard, the \textit{Scrum Team} must create a Definition of Done appropriate for the product
	\item 
\end{itemize}

\nocite{*}
\bibliographystyle{plain}
\bibliography{bibliography}

\end{document}